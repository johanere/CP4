%%
%% Automatically generated file from DocOnce source
%% (https://github.com/hplgit/doconce/)
%%
%%


%-------------------- begin preamble ----------------------

\documentclass[%
oneside,                 % oneside: electronic viewing, twoside: printing
final,                   % draft: marks overfull hboxes, figures with paths
10pt]{article}

\listfiles               %  print all files needed to compile this document

\usepackage{relsize,makeidx,color,setspace,amsmath,amsfonts,amssymb}
\usepackage[table]{xcolor}
\usepackage{bm,ltablex,microtype}

\usepackage[pdftex]{graphicx}

\usepackage{fancyvrb} % packages needed for verbatim environments

\usepackage[T1]{fontenc}
%\usepackage[latin1]{inputenc}
\usepackage{ucs}
\usepackage[utf8x]{inputenc}

\usepackage{lmodern}         % Latin Modern fonts derived from Computer Modern

% Hyperlinks in PDF:
\definecolor{linkcolor}{rgb}{0,0,0.4}
\usepackage{hyperref}
\hypersetup{
    breaklinks=true,
    colorlinks=true,
    linkcolor=linkcolor,
    urlcolor=linkcolor,
    citecolor=black,
    filecolor=black,
    %filecolor=blue,
    pdfmenubar=true,
    pdftoolbar=true,
    bookmarksdepth=3   % Uncomment (and tweak) for PDF bookmarks with more levels than the TOC
    }
%\hyperbaseurl{}   % hyperlinks are relative to this root

\setcounter{tocdepth}{2}  % levels in table of contents

% --- fancyhdr package for fancy headers ---
\usepackage{fancyhdr}
\fancyhf{} % sets both header and footer to nothing
\renewcommand{\headrulewidth}{0pt}
\fancyfoot[LE,RO]{\thepage}
% Ensure copyright on titlepage (article style) and chapter pages (book style)
\fancypagestyle{plain}{
  \fancyhf{}
  \fancyfoot[C]{{\footnotesize \copyright\ 1999-2018, "Computational Physics I FYS3150/FYS4150":"http://www.uio.no/studier/emner/matnat/fys/FYS3150/index-eng.html". Released under CC Attribution-NonCommercial 4.0 license}}
%  \renewcommand{\footrulewidth}{0mm}
  \renewcommand{\headrulewidth}{0mm}
}
% Ensure copyright on titlepages with \thispagestyle{empty}
\fancypagestyle{empty}{
  \fancyhf{}
  \fancyfoot[C]{{\footnotesize \copyright\ 1999-2018, "Computational Physics I FYS3150/FYS4150":"http://www.uio.no/studier/emner/matnat/fys/FYS3150/index-eng.html". Released under CC Attribution-NonCommercial 4.0 license}}
  \renewcommand{\footrulewidth}{0mm}
  \renewcommand{\headrulewidth}{0mm}
}

\pagestyle{fancy}


% prevent orhpans and widows
\clubpenalty = 10000
\widowpenalty = 10000

% --- end of standard preamble for documents ---


% insert custom LaTeX commands...

\raggedbottom
\makeindex
\usepackage[totoc]{idxlayout}   % for index in the toc
\usepackage[nottoc]{tocbibind}  % for references/bibliography in the toc
\usepackage{listings}
\usepackage{verbatim}
%-------------------- end preamble ----------------------

\begin{document}

% matching end for #ifdef PREAMBLE

\newcommand{\exercisesection}[1]{\subsection*{#1}}


% ------------------- main content ----------------------



% ----------------- title -------------------------

\thispagestyle{empty}

\begin{center}
{\LARGE\bf
\begin{spacing}{1.25}
Project 4
\end{spacing}
}
\end{center}

% ----------------- author(s) -------------------------

\begin{center}
{\bf \href{{http://www.uio.no/studier/emner/matnat/fys/FYS3150/index-eng.html}}{Computational Physics I FYS3150/FYS4150}}
\end{center}

    \begin{center}
% List of all institutions:
\centerline{{\small Department of Physics, University of Oslo, Norway}}
\end{center}
    
% ----------------- end author(s) -------------------------

% --- begin date ---
\begin{center}
Aug 6, 2018
\end{center}
% --- end date ---

\vspace{1cm}
The Stern Gerlach experiment in 
\section{Methods}

\subsection{Magnetism and spin}
In general, magnets are object which produce a magnetic field. The source of such fields are net magnetic moment and induction from electric currents. Net magnetic moment arises when the electron spins and orbital motions do not balance out \cite{feynan}.

In ferromagnetic materials, the

\begin{comment}
Assume knowledge of basic statistical quantities (<x>)
Assume knowledge of thermodynamic quantities 

\section{Methods}
- Brief description of magnets (finn en kilde)
	- magnetic properties in relation to spin- stern gerlach -> sammenheng mellom spinn og magnetisme (ikke grei ut mye)
	- Sammenheng mellom temp og magnetisk egenskaper (M) -> faseovergang
	- Energy relatert til SS av systemet - stabilty->ss. Helmholtz free. Ser på magnet om isolert system -> må ha med varmekap.
	-critical temp on magnets - lose magn. properties in phase 	 		transition ( exact critical temp Lars Onsager)
	- mål på faseovergang: varmekap, mag. susc, chi (reponse of magnet to ext. mag field) 
	
- Describe Ising model in general
	- Lattice for å sette spin i system (slik at det er et objekt, og ikke bare partikler med spinn)
	- P.B.C - torus
	- $|\bar{M}|=|M|$ gjnsnitt. av magnetic moment fra hvert spinn
	- Energy
	- $C_V$
	- $\chi$
	- Ta med math disc av <E> etc.
	
Assume two spins in each dimension, $L=2$. Find analytical expression for $Z$, and corresponding $<E>$, $|M|$, $C_v$ and susceptibility $\chi$ as functions of $T$ using periodic boundary conditions (P.B.C). IOT get benchmark calculations for rest of paper. \newline

*0 analytical sols for $L=2$: benchmarks

\subsection{Simulating energy and magnetization using Ising model}

Discretization gjøres ved ising, siden den bruker lattice. Tid er antall mccycles (lattice sweep)
Methods used in num. sim, MCMC using metrop.

- Describe MCMC - bruker en MCMCmetode til å simulere endringer i microtilt. modellert ved Ising 
	- PDF
	- M.Chains?
	- Metropolis 
		- garanterer detailed balance, ergodicity (alle tilst. tilgjenglig)
	- RNG
	- MC cycles
	
- Develop Ising model algo using MC 
	- MC
	- Metrop
	- Must calc: $\bar{E}$
	- ikke beskriv i detalj feil, flops etc.
	
- Implement algo
	- Construction, take L, Einit? spin config?
	- Count cycles
	- Plot energy as func of #MCcycles
	- Plot $\bar{E}$
		
Write code for Ising model, which computes mean energy ($<E>$ ?), $|M|$, $C_v$, and $\chi$ as functions of $T$ using P.B.C for $L=2$. Compare with analytical results for $T=1.0 [kT/J]$. How many cycles needed for good (max(abserr)<1e-4?) agreement? \newline

Using $L=20$. Study number of MCcycles needed iOT reach most likely state. Plot various expectation values as function of #MCcycles, using ($T=1.$ and $T=2.4$). MCcycles rep time. #MCc needed to reach eq.state? Est. time to reach eq. state -> eq.time. Plot #accepted configs (microstates?) -> how do #configss behave in rel. to T?
*1 Find # states needed to reach SS. 


- Mod prog to log state in each cycle
Compare $P(E)$ for $L=20$ and $T=1,2.4$. $P(E)$ found by counting #times a given energy appears in computation. Use result *1. Compare PE with variance in energy discuss behaviour.


\subsection{Simulating Phase transitions} 
- Describe Ising model near phase transitions (near $T_C$).

- Mod prog to plot Cv, $\chi$
\section{Results}


\end{comment}



% ------------------- end of main content ---------------

\end{document}

